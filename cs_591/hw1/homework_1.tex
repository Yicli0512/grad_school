%% to produce a PDF copy, issue the following command:
%%
%%     pdflatex propositional-logic-examples.tex
%%
%% in the same directory containing the LaTeX style files:
%%
%%     prooftree.sty  and  boxproof.sty

\documentclass[11pt,leqno,fleqn]{article}

\usepackage{graphicx} 
\usepackage{times}              % better fonts for mathematical symbols
\usepackage{bm}                 % unlike \boldmath,

                                % \bm can be used anywhere within math mode
                                
\usepackage{amsfonts}
\usepackage{amsmath}                               
\usepackage[scaled=0.9]{helvet} % makes text a little smaller throughout,
                                % but not the text in math mode.


\setlength\hoffset{-5pt}      % horizontal offset, to move text horizontally
\setlength{\textwidth}{4.5in} % try different widths
\setlength\voffset{-5pt}      % vertical offset, to move text vertically
\setlength{\textheight}{7in}  % try different heights

\newcommand{\Hide}[1]{}             % use \Hide{bla bla} to hide ``bla bla''
\newcommand{\code}[1]{\texttt{#1}}  % use \code{...} to produce ASCII chars

\newcommand{\Cov}{\mathrm{Cov}}


\begin{document}

\section{}
I have uploaded an excel spredsheet, that contains the harmonics, aliased frequencies, the absolute value of those frequencies, and them all sorted.
\section{}
a.\\
$\omega = 2 \pi f = 2 * \pi * 261.6 = 1643.68$\\
b.\\
From equation 2.55 in the book $sin(\theta) =-i \frac{e^{i \theta} - e^{-i \theta}}{2}$\\
So in our case:\\
Let $\phi = 2 * \pi * 261.6  * i/rate$ (where i is the iteration in the for loop)\\
Then $sin(\phi) = -i \frac{e^{i \phi} - e^{-i \phi}}{2}$ would be expression for the phasor representing middle C.
\section{}
SNR - Signal to Noise Ratio\\
DR- Dynamic Range\\
\\
64 bit Integer:\\
$SNR = DR = 20\ log_{10} (2^{64}) = 385.31$\\
\\
64 Bit Floating Point\\
$DR = 6.02 * 2^{11} = 12328.96$\\
$SNR = 6.02 * {52} = 313.04$\\
\section{}
a. \\
$\frac{-4 + i}{-3 +2i} = \frac{-4 + i}{-3 +2i} * \frac{-3 - 2i}{-3 - 2i}  = \frac{12 + 8i -3i -2i^2}{13} = \frac{14 + 5i}{13}$\\
\\
b.1.\\
$(i+1)^6 = ((i+1)^2)^3 = (1 + 2i + i^2)^3 = (2i)^3 = -8i$\\
Absolute Value: $8$\\
Complex Conjugate: $8i$\\
\\
b.2.\\
$i^{17} = i^{16} i^1 = (i^4)^4i = i$\\
Absolute Value: $1$\\
Complex Conjugate: $-i$\\
\\
\\
c.\\
$i^5 + i + 1 = i^4 i + i + 1 = 1 + 2 i$\\
\\
d.1\\
$8 =  8( cos(2 \pi k) + isin(2 \pi k) ) , k \in \mathbb{Z}$\\
\\
d.2\\
$6 =  6( cos( \frac{\pi}{2}  k) + isin( \frac{\pi}{2} k) ) , k \in \mathbb{Z}$\\
\end{document}

