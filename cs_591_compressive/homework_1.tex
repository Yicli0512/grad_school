%% to produce a PDF copy, issue the following command:
%%
%%     pdflatex propositional-logic-examples.tex
%%
%% in the same directory containing the LaTeX style files:
%%
%%     prooftree.sty  and  boxproof.sty

\documentclass[11pt,leqno,fleqn]{article}

\newcommand{\tab}{\hspace*{2em}}
\usepackage{amsfonts}
\usepackage{fullpage, enumerate}
\usepackage{amsmath}
\usepackage{amssymb}
\usepackage{qtree}
\usepackage{listings}

\usepackage{graphicx} 
\usepackage{times}              % better fonts for mathematical symbols
\usepackage{bm}                 % unlike \boldmath,
                                % \bm can be used anywhere within math mode
\usepackage[scaled=0.9]{helvet} % makes text a little smaller throughout,
                                % but not the text in math mode.
%\usepackage{./tex/latex/misc/prooftree}
%\usepackage{./tex/latex/misc/boxproof}

\setlength\hoffset{-5pt}      % horizontal offset, to move text horizontally
\setlength{\textwidth}{4.5in} % try different widths
\setlength\voffset{-5pt}      % vertical offset, to move text vertically
\setlength{\textheight}{7in}  % try different heights

\newcommand{\Hide}[1]{}             % use \Hide{bla bla} to hide ``bla bla''
\newcommand{\code}[1]{\texttt{#1}}  % use \code{...} to produce ASCII chars
\newcommand{\Intro}[1]{{#1}{\textrm{i}}}
\newcommand{\Elim}[1]{{#1}{\textrm{e}}}


\newcommand{\norm}[1]{\left\lVert#1\right\rVert}

\newcommand{\matrixA}{ $ 
\begin{bmatrix}
  a_{11} & ... & a_{1N} \\
  . & . & . \\
  . & . & .  \\
   a_{m1} & ... & a_{mN}
 \end{bmatrix}
 $}

\newcommand{\vectorX}{ $ 
\begin{bmatrix}
  x_{1} \\
  .  \\
  .   \\
  x_{N}
 \end{bmatrix}
 $}
 
 \newcommand{\vectorAx}{  
\begin{bmatrix}
  a_{11}x_1 + ... + a_{1N}x_N \\
  .  \\

  .   \\
 a_{m1}x_1 + ... + a_{mN}x_N
 \end{bmatrix}
 }

\title{CS 591, Fall 2014
       \\[1ex]
       \textbf{Assignment 1}}
\author{Shan Sikdar} 
\date{} % omit date

\begin{document}

\maketitle

\begin{enumerate}[(1)]
% (1)
\item Recall from class that for a matrix $A = (a_{jk} \in C^{m x N})$, the operator norm of A from $l_p$ into $l_p$ is defined as:
$|| A ||_{p \to p} := max_{||x||_p = 1} ||Ax||_p$
\begin{enumerate}[(a)]
%(a)
 \item $|| A ||_{1 \to 1} = max_{k \in [N]}\sum_j^m |a_{jk}|$\\
 proof.\\
 If  A = \matrixA, and x = \vectorX, then $Ax = \vectorAx$\\ \\
 Therefore:\\ \\
$\norm{A}_{1 \to 1} = max_{||x||_1 = 1} \norm{Ax}  =  max_{||x||_1 = 1} \norm{  \vectorAx }_1$ \\ \\ \\
$ = max_{||x||_1 = 1}   \ |a_{11}x_1 + ... + a_{1N}x_N| + ... + |a_{m1}x_1 + ... + a_{mN}x_N|  $\\ 

Since this is the max over all $||x||_1 = 1$, this would happen when $x_i = 1$  for some index i and 0 for all other indices. \\
The result would then be $\sum_{j=1}^m |a_{jk}|$ for an index k. We would have to go through each index k from 1 to N to see which would give the 
greatest result for the maximum.  Then:\\
$ max_{||x||_1 = 1}   \ |a_{11}x_1 + ... + a_{1N}x_N| + ... + |a_{m1}x_1 + ... + a_{mN}x_N|  $\\  \\
$ = max_{k \in [N]} \sum_{j=1}^m |a_{jk}| $

 \newpage
\item $|| A ||_{\infty \to \infty} = max_{j \in [m]}\sum_{k = 0}^N |a_{jk}|$\\ \\
proof.\\ \\
Using A, x and Ax as defined in part (a):\\ \\
$\norm{A}_{\infty \to \infty} = max_{||x||_{\infty} = 1} \norm{Ax}_{\infty} =  max_{||x||_{\infty} = 1} \norm{  \vectorAx }_{\infty} $ \\ \\ \\
$= max_{||x||_{\infty} = 1} max(| a_{11}x_1 + ... + a_{1N}x_N|, ... , |a_{m1}x_1 + ... a_{mN}x_N| ) $\\

Since this is the max over all $||x||_{\infty} = 1$, this will happen when we take x to be a vector made up of 1's and -1's. \\ \\
$=  max(|a_{11}| + ... + |a_{1N}|, ... , |a_{m1}| + ... + |a_{mN}| ) $\\ \\
$ = max(\sum_{k =1}^N |a_{1k}|, ... , \sum_{k =1}^N |a_{mk}|) $ \\ \\
So the result is the maximum of the L1 norms of every row:\\
$ = max_{j \in [m]} \sum_{k =1}^N |ajk|$

\newpage
\item $\norm{A}_{2 \to 2} = \sigma_{max}(A) = \sqrt{\lambda_{max}(A^*A)}$ where $ \sigma_{max}(A)$ denotes the largest singular value
of A and $ \sqrt{\lambda_{max}(A^*A)}$ is the largest eigenvalue of $A^*A$.\\
proof.\\ \\
Let $\Gamma = A^*A$, let $\lambda_1 .... \lambda_n$ be the eigenvalues of $\Gamma$. \\
Let ${e_1,...e_n}$ be the orthonormal basis of the Euclidian Space.\\
Let $v \in C^N$ be $v = a_1 e_1 + ... + a_n e_n = \sum_i^n a_i e_i$. So that: \\
 $\norm{v}^2 = |<v,v>| = |<\sum_i^n a_i e_i, \sum_i^n a_i e_i>| = \sum_i^n a_i ^2$ \\ \\
 Now look at $\Gamma v$:\\
 $\Gamma v = \Gamma (\sum_i^n a_i e_i) = \sum_i a_i \Gamma(e_i) = \sum a_i \lambda_i e_i$. \\
 Denote the largest $\lambda_i$ value to be $\lambda_{max}$.\\ \\
 Now want to show $\norm{A} \leq \sqrt{\lambda_{max}(A^*A)}$ ,\\
 $\norm{Av}^2 = <Av,Av>$\\
 $= (Av)^T (Av) = v^T A^T Av = <v, A^T A v>$\\
 $= <v, \Gamma v> = <\sum a_i e_i, \sum \lambda_i a_i e_i>$\\
$= \sum (a_i e_i)(\lambda_i a_i e_i)^T = \sum a_i e_i e_i^T a_i^T \lambda_i^T$\\
$ = \sum a_i a_i^T \lambda_i^T \leq \sum a_i a_i^T \lambda_{max} = \lambda_{max} \norm{v}^2$\\
So $\norm{A}^2 \leq \lambda_{max} $. \\
So $\norm{A} \leq \sqrt{\lambda_{max}(A^*A)}$\\ \\

Now want to show $\norm{A} \geq \sqrt{\lambda_{max}(A^*A)}$, \\
Consider $x_0 = e_{max}$. $\norm{x} = 1$. Where $e_{max}$ is the eigenvector corresponding to $\lambda_{max}$ .\\
$<x,\Gamma x> = <e_{max}, \Gamma(e_{max})> = <e_{max}, \lambda_{j_0} e_{max}> = |\lambda_{max}| $.
But $<x_0, \Gamma x_0> \leq \norm{A}^2$ \\ \\
So  $\norm{A} \geq \sqrt{\lambda_{max}(A^*A)}$.\\ 

If  $\norm{A} \geq \sqrt{\lambda_{max}(A^*A)}$ and  $\norm{A} \leq \sqrt{\lambda_{max}(A^*A)}$,\\ \\
Then  $\norm{A} = \sqrt{\lambda_{max}(A^*A)}$


	
 \end{enumerate}
 
 \item 
 Given a 
 
 \item Let V be a normed vector space with an inner product. Prove Cauchy Shqartz Inequality. \\
 Proof.\\ \\
 Let $u,v \in V$. If v = 0 then the equality holds trivially. So assume $v \neq 0$. \\
 Look at its orthogonal decomposition.\\
 $u = \frac{<u,v>}{\norm{v}^2} v + (u - \frac{<u,v>}{\norm{v}^2}v)$\\
 $= \frac{<u,v>}{\norm{v}^2} v + w$ here w is orthognal to v.\\
 Using Pythgorean Theorem:\\
 $\norm{u}^2 = \norm{\frac{<u,v>}{\norm{v}^2} v}^2 + \norm{w}^2$\\
 $\geq \frac{|<u,v>|^2}{\norm{v}^2}$.\\ \\ \\
 So:
 $\norm{u}^2 \geq \frac{|<u,v>|^2}{\norm{v}^2}$\\ \\
 Then: $ \norm{v}^2 \norm{u}^2 \geq |<u,v>|^2 $\\ \\
 Then:  $ \norm{v} \norm{u} \geq |<u,v>| $
  
  \end{enumerate}
 
 
\end{document}

