%% to produce a PDF copy, issue the following command:
%%
%%     pdflatex propositional-logic-examples.tex
%%
%% in the same directory containing the LaTeX style files:
%%
%%     prooftree.sty  and  boxproof.sty

\documentclass[11pt,leqno,fleqn]{article}

\usepackage{graphicx} 
\usepackage{times}              % better fonts for mathematical symbols
\usepackage{bm}                 % unlike \boldmath,
                                % \bm can be used anywhere within math mode
\usepackage[scaled=0.9]{helvet} % makes text a little smaller throughout,
                                % but not the text in math mode.
\usepackage{./tex/latex/misc/prooftree}
\usepackage{./tex/latex/misc/boxproof}

\setlength\hoffset{-5pt}      % horizontal offset, to move text horizontally
\setlength{\textwidth}{4.5in} % try different widths
\setlength\voffset{-5pt}      % vertical offset, to move text vertically
\setlength{\textheight}{7in}  % try different heights

\newcommand{\Hide}[1]{}             % use \Hide{bla bla} to hide ``bla bla''
\newcommand{\code}[1]{\texttt{#1}}  % use \code{...} to produce ASCII chars
\newcommand{\Intro}[1]{{#1}{\textrm{i}}}
\newcommand{\Elim}[1]{{#1}{\textrm{e}}}

\title{CS 512, Spring 2014
       \\[1ex]
       \textbf{Assignment 1}}
\author{Shan Sikdar} 
\date{Due Monday January 27th} % omit date

\begin{document}

\maketitle

\section{Problem 1(a): $P\to Q,  P \to \neg{Q}\vdash \neg{P}$}

\begin{proofbox}
   \label{a1}\: P\to Q \= \textrm{premise} \\
   \label{a2}\: P\to \neg{Q} \= \textrm{premise} \\
   \[
      \label{a3}\: P  \= \textrm{assume} \\
      \label{a4}\: Q    \= \Elim{\to}\ \ref{a1},\ref{a3} \\
       \label{a5}\: \neg{Q}    \= \Elim{\to}\ \ref{a2},\ref{a3} \\
        \label{a6}\: \bot \= {\Elim{\neg{}}}\ \ \ref{a4},\ref{a5}\\
   \]
     \label{a7}\: \neg{P} \= \Intro{\neg{}} \\
\end{proofbox}

\section{Problem 1(b): $P\to(Q\to R),P,\neg{R} \vdash \neg{Q}$}

\begin{proofbox}
   \label{b1}\: P\to(Q\to R) \= \textrm{premise} \\
    \label{b2}\: P \= \textrm{premise} \\
     \label{b3}\: \neg{R} \= \textrm{premise} \\
   \[
      \label{b4}\: Q    \= \textrm{assume} \\
      \label{b5}\:  Q\to R \= \Elim{\to}\ \ \ref{b1},\ref{b2}\\
      \label{b6}\:  R \= \Elim{\to}\ \ \ref{b4},\ref{b5}\\
      \label{b7}\: \bot \= {\Elim{\neg{}}}\ \ \ref{b3},\ref{b6}\\
   \]
       \label{b8}\: \neg{Q} \= \Intro{\neg{}} \\
\end{proofbox}

\section{Problem 2: 1.2.2(g): $p \land \neg{p} \vdash \neg{(r \to q)} \land (r \to q)$}
\begin{proofbox}
   \label{c1}\: p\land \neg{p} \= \textrm{premise} \\
    \label{c2}\: p \= \Elim{\land}_1\ \ \ref{c1} \\
     \label{c3}\: \neg{p} \= \Elim{\land}_1 \ \ \ref{c1} \\
      \label{c4}\: \bot \= {\Elim{\neg{}}}\ \ \ref{c2},\ref{c3} \\
      \label{}\: \neg{(r \to q)} \land (r \to q) \= \Elim{\bot} \\
\end{proofbox}

\section{Problem 2: 1.2.2(h): $p \to q, s\to t \vdash p \lor s \to q \land t$}
 Since in the truth table some valuations have the left evaluate to T while the result false,
 the sequent is not valid.


\newcommand{\TTT}{\bm{\mathsf{T}}}
\newcommand{\FFF}{\bm{\mathsf{F}}}


\[
\begin{array}{ c | c | c | c | c | c | c | c | c }
 p & q & s &  t
    & p \to q & s\to t  & p \lor s &  q \land t &  p \lor s \to q \land t

% s: TRUE
% t: TRUE
\\ \hline 
 \TTT & \TTT & \TTT & \TTT & \TTT & \TTT &  \TTT & \TTT & \TTT
\\ \hline  
 \TTT & \FFF & \TTT & \TTT & \FFF  & \TTT  &  \TTT  & \FFF & \FFF
\\ \hline  
 \FFF & \TTT & \TTT & \TTT & \TTT  & \TTT  &   \TTT & \TTT & \TTT
\\ \hline  
 \FFF & \FFF & \TTT & \TTT & \TTT  & \TTT &     \TTT & \FFF & \FFF

% s: TRUE
% t: FALSE
\\ \hline 
 \TTT & \TTT & \TTT & \FFF & \TTT  & \FFF  &  \TTT & \FFF & \FFF
\\ \hline  
 \TTT & \FFF & \TTT & \FFF & \FFF  &  \FFF &  \TTT   & \FFF  & \FFF
\\ \hline  
 \FFF & \TTT & \TTT & \FFF & \TTT & \FFF &     \TTT & \FFF & \FFF
\\ \hline  
 \FFF & \FFF & \TTT & \FFF & \TTT & \FFF  &     \TTT & \FFF & \FFF


% s: FALSE
% t: TRUE
\\ \hline 
 \TTT & \TTT & \FFF & \TTT & \TTT  & \TTT  &  \TTT  & \TTT & \TTT
\\ \hline  
 \TTT & \FFF & \FFF & \TTT & \FFF  &  \TTT &  \TTT    & \FFF & \FFF
\\ \hline  
 \FFF & \TTT & \FFF & \TTT & \TTT  & \TTT  &  \FFF  &  \TTT & \TTT
\\ \hline  
 \FFF & \FFF & \FFF & \TTT & \TTT  & \TTT  &  \FFF   & \FFF & \FFF

% s: FALSE
% t: FALSE
\\ \hline 
 \TTT & \TTT & \FFF & \FFF & \TTT  & \TTT  & \TTT  & \FFF & \FFF
\\ \hline  
 \TTT & \FFF & \FFF & \FFF & \FFF  & \TTT  &  \TTT   & \FFF & \FFF
\\ \hline  
 \FFF & \TTT & \FFF & \FFF & \TTT  &  \TTT &  \FFF   & \FFF &  \TTT
\\ \hline  
 \FFF & \FFF & \FFF & \FFF & \TTT  & \TTT  & \FFF & \FFF &  \TTT


\end{array}
\]



\section{Problem 2: 1.2.2(i): $ \neg{(\neg{p} \lor q)} \vdash p $}

\begin{proofbox}
	\label{e1}\: \neg{(\neg{p} \lor q)}  \=  \textrm{premise}\\
	  \[
      \label{e2}\: \neg{p}    \= \textrm{assume} \\
      \label{e3}\: \neg{p} \lor q \= \Intro{\lor}\ \ \ref{e2}\\
      \label{e4}\: \bot \= {\Elim{\neg{}}}\ \ \ref{e1},\ref{e3}\\
   \]
	\label{e5}\: p \= \Intro{\neg{}}\\
\end{proofbox}

\section{Problem 3:}

 $  S -1 = C$ \\
Justification:
For every propostional atom you add to a well formed formula, you need a binary connective to join it to.\\

$N \leq S + C $\\

This inequality assumes that any series of negation signs can be reduced to one or no negation signs (e.g. $\neg{\neg{\neg{q}}} = \neg{q} $ or $\neg{\neg{\neg{\neg{q}}}} = q$) . Otherwise there is no upperbound for the value of N

\section{Problem 4: 1.4.12(a):}
If $p$ is True, $q$ is True, then $\neg{p} \lor (q \to p) $ is True while $\neg{p} \land q$ is False. So sequent not valid.

\section{Problem 4: 1.4.12(b):}
If r is True , q is False, and p is true, then $\neg{r} \to (p \lor q)$ is True, and $r \land \neg{q}$ is True. But $r \to q$ is False. So sequent not valid.

\section{Problem 4: 1.4.12(c):}
If p is True, q is False, and r is True, then $p \to (q \to r)$ is True. But $p \to (r \to q)$ is false. So sequent not valid.

\newpage
\section{problem 5:}

Use induction based on the height of the parse tree formed by the WFF $\varphi$.\\

Base Case: Height = 1,\\
If Height = 1, then the parse tree must only conisist of propsitional atom, say p. Then $\varphi  = p$ and $\varphi * = \neg{p} = \neg{\varphi}$.\\

Inductive Step: Assume true for all WFF's of height  n. Prove for height n + 1.\\ \\
Let $\varphi$ be a WFF with height n+1. For $\varphi$ to have this hieght one of the following cases must happen:\\
1. $\varphi$ is made up of another WWF $\phi$ of parse tree height  n with a $\neg{}$ in front of it: ( $\varphi = \neg{\phi}$)\\
2. $\varphi$ is made up of two WFF's $\phi,\rho$ of parse tree height n connected by $\land$: ($\varphi = \phi \land \rho$) \\
3. $\varphi$ is made up of two WFF's $\phi,\rho$ of parse tree height n connected by $\lor$: ($\varphi = \phi \lor \rho$) \\
\\
Case 1: $\varphi = \neg{\phi}$ \\
  Using the inductive hypothesis we know $\phi * $ is tuautologically equivelent to $\neg{\phi}$. So $\varphi = \phi * = \neg{\phi}$ \\
 Then: \\
 $\varphi * =  \neg{(\neg{\phi})}$\\
$\varphi * =  \neg{(\phi *)}$\\
$\varphi * =  \neg{(\varphi )}$\\
$\varphi * =  \neg{\varphi }$\\
 
Case 2:  $\varphi = \phi \land \rho $ . Using the inductive hypothesis we know $\neg{\phi} = \phi *$ and $\neg{\rho} = \rho *$\\
Then because:\\
$ \neg{\varphi}  = \neg{(\phi \land \rho)}$ and $\varphi * =  \neg{\phi} \lor \neg{\rho}  =  \phi * \lor \rho* $ . \\
I can use a truth table to show $\varphi *$ and $ \neg{\varphi}$ tautological equivlence: 

\[
\begin{array}{ c | c | c | c | c | c | c | c | c }
 \phi & \rho & \varphi = \phi \land \rho &  \neg{\varphi} & \neg{\phi} & \neg{\rho}&\varphi * = \neg{\phi} \lor \neg{\rho} 
\\ \hline 
 \TTT & \TTT & \TTT & \FFF & \FFF  & \FFF & \FFF
\\ \hline  
 \TTT & \FFF &  \FFF & \TTT  & \FFF & \TTT & \TTT
 \\ \hline  
 \FFF & \TTT & \FFF & \TTT & \TTT & \FFF & \TTT
 \\ \hline  
 \FFF & \FFF & \FFF  & \TTT  & \TTT &  \TTT & \TTT
\end{array}
\]

The $\varphi * $ column has the same values as the $\neg{\varphi}$ column. So they are tautologically equivelent \\

Case 3: $\varphi = \phi \lor \rho$. Again  using the inductive hypothesis, $\neg{\phi} = \phi *$ and $\neg{\rho} = \rho *$\\
Then because:\\
$ \neg{\varphi}  = \neg{(\phi \lor \rho)}$ and $\varphi * =  \neg{\phi} \land \neg{\rho}  =  \phi * \land \rho* $ . \\
I can use a truth table to show $\varphi *$ and $ \neg{\varphi}$ tautological equivlence: 

\[
\begin{array}{ c | c | c | c | c | c | c | c | c }
 \phi & \rho & \varphi = \phi \lor \rho &  \neg{\varphi} & \neg{\phi} & \neg{\rho}&\varphi * = \neg{\phi} \land \neg{\rho} 
\\ \hline 
 \TTT & \TTT & \TTT & \FFF & \FFF  & \FFF & \FFF
\\ \hline  
 \TTT & \FFF &  \TTT & \FFF  & \FFF & \TTT & \FFF
 \\ \hline  
 \FFF & \TTT & \TTT & \FFF & \TTT & \FFF & \FFF
 \\ \hline  
 \FFF & \FFF & \FFF  & \TTT  & \TTT &  \TTT & \TTT
\end{array}
\]

The $\varphi * $ column has the same values as the $\neg{\varphi}$ column. So they are tautologically equivelent \\

\end{document}

