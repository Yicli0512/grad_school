%% to produce a PDF copy, issue the following command:
%%
%%     pdflatex propositional-logic-examples.tex
%%
%% in the same directory containing the LaTeX style files:
%%
%%     prooftree.sty  and  boxproof.sty

\documentclass[11pt,leqno,fleqn]{article}

\newcommand{\tab}{\hspace*{2em}}
\usepackage{amsfonts}
\usepackage{fullpage, enumerate}
\usepackage{amsmath}
\usepackage{amssymb}
\usepackage{qtree}
\usepackage{listings}

\usepackage{graphicx} 
\usepackage{times}              % better fonts for mathematical symbols
\usepackage{bm}                 % unlike \boldmath,
                                % \bm can be used anywhere within math mode
\usepackage[scaled=0.9]{helvet} % makes text a little smaller throughout,
                                % but not the text in math mode.
%\usepackage{./tex/latex/misc/prooftree}
%\usepackage{./tex/latex/misc/boxproof}

\setlength\hoffset{-5pt}      % horizontal offset, to move text horizontally
\setlength{\textwidth}{4.5in} % try different widths
\setlength\voffset{-5pt}      % vertical offset, to move text vertically
\setlength{\textheight}{7in}  % try different heights

\newcommand{\Hide}[1]{}             % use \Hide{bla bla} to hide ``bla bla''
\newcommand{\code}[1]{\texttt{#1}}  % use \code{...} to produce ASCII chars
\newcommand{\Intro}[1]{{#1}{\textrm{i}}}
\newcommand{\Elim}[1]{{#1}{\textrm{e}}}

\newcommand{\TTT}{\bm{\mathsf{T}}}
\newcommand{\FFF}{\bm{\mathsf{F}}}

\title{CS 512, Spring 2014
       \\[1ex]
       \textbf{Assignment 6}}
\author{Shan Sikdar} 
\date{} % omit date

\begin{document}

\maketitle

\section{Problem 1}
\begin{enumerate}[(a)]
% (a)
\item Apply the translation of the proceeding formula's to four wwf's of ``First-Order Mondaic Logic of Linear Order'':
\begin{enumerate}[(1)]
%(1)
 \item $\varphi_1 \equiv p \bold{U} q$\\
 $\exists \ t' \ [t' \geq t \land q(t') \land \forall \ t'' \ [t \leq t'' \leq t' \to p(t'')]]$
 
 %(2)
 \item $\varphi_2 \equiv \bold{W} q \land \bold{F}q$\\
  %def of W
  $\exists \ t' \ [t' \geq t \land q(t') \land \forall \ t'' \ [t \leq t'' \leq t' \to p(t'')]] \lor \forall t''' (t''' \geq t \to  p(t))$
  %def of AND F
  $\land \exists t^{IV} ( t''''\geq t \land q(t''''))$
 %(3)
 \item $p \bold{W} q$\\
 %def of W
  $\exists \ t' \ [t' \geq t \land q(t') \land \forall \ t'' \ [t \leq t'' \leq t' \to p(t'')]] \lor \forall t'''  (t''' \geq t \to p(t'''))$
 %(4)
 \item $p \bold{U} q \lor \bold{G}p$\\
  $\exists \ t' \ [t' \geq t \land q(t') \land \forall \ t'' \ [t \leq t'' \leq t' \to p(t'')]] \ \ \lor \ \ \forall t' [t' \geq t \to p(t')] $
 
 \end{enumerate}
 \item For $\psi_1$ and $\psi_2$ to be equivelent they will be able to imply each other in first oder monadic logic. So $\psi_1 \to \psi_2$ as well as $\psi_2 \to \psi_1$.  $\psi_1$ and $\psi_2$ will also be equivelent if they contain the same semantic models, so that they would have the same truth values under any interpretation. Thus for any t and other variables that satisfy $\phi_1$, this arrangement will also lead to $\psi_2$ being satisfied. (Similar Reasoning for $\psi_3$ and $\psi_4$)
 \\For WFF's containing free varibles, free variables are implicitly universally quantified, so universal quantifiers over these variables can be added, resulting in a formula with no free variables. (source wikipedia). 

 \item You could use the deductive system of the first order logic(we did this in class) to prove the equivelence of $\psi_1$ and $\psi_2$  ($\psi3$ and $\psi 4$ as well)
  We could also show that models of $\psi_1$ are models of $\psi_2$, and the show that the models of $\psi_2$ are models of $\psi_1$.
  We could also convert these formulas back to Linear Temporal Logic and show equivelence in that logic as well.
 According to wikipedia: Any deductive system of first order logic can be used to prove the equivelence. One such one is  The Tableux Method. Other include  Sequent Calculus, Hilbert deductive system, and resolution.
 So these are some of the multiple ways to show equivelence.
\end{enumerate}

\newpage
\section{Problem 2}
\begin{enumerate}[(a)]

%(a)
\item Precisely specify the paths of M that saitsfy $\varphi_{fair}$
	\begin{enumerate}[(1)]

		\item 	
		$\omega$-regular expression:\\
		$(s_0 s_1)^{\omega} + s_0 s_1 (s_0 s_1)^* s_2^{\omega} + s_3 s_3^* s_4 s_5^{\omega}$\\ \\
	Note that $\varphi_{fair} \triangleq (\bold{GF}(p \land q)\to \bold{GF}\neg r) \land (\bold{FG}(p \land q) \to \bold{FG}\neg q)$\\
	can be reduced to: \\
	$(\bold{FG}(p \to q) \lor \bold{GF} \neg r) \land (\bold{GF}(\neg p \lor \neg q) \lor \bold{GF \neg q})$\\
	and I used this form to see which paths satisfied the fairness statement.

		
	\end{enumerate}

%(b)
\item 
	\begin{enumerate}[(1)]
	
		\item $\psi_1 \triangleq \bold{X}\neg p \to \bold{FG}p \equiv \neg \bold{X}\neg p \lor  \bold{FG}p \equiv \bold{X} p \lor  \bold{FG}p  $\\
		$\pi =(s_0 s_1)^{\omega}$\\
		This path satisfies the fairness statement but not $\psi_1$ since when the path starts $\bold{X}p$ is false and $\bold{FG}p$ will never be true.
	
		\item $\psi_2 \triangleq q \bold{ \ U \ G} \neg q   $\\
		$\pi =(s_0 s_1)^{\omega}$\\
		This path satisfies the fairness statement but not $\psi_2$ since $\neg q$ never becomes true
		\item $\mathcal{M} \models_{fair} \psi_3$\\
		All paths that satisfy the fairness condiiton satisfy the conditions for weak until.
	\end{enumerate}
\end{enumerate}

\newpage
\section{Problem 3}
	\begin{enumerate}[(a)]
	\item $[|\top|] = S$\\
	By definition the statisfaction of a WFF of CTL is defined if for all $s \in S$, $\mu, s \models \top$. So this follows straight from definition.
	\item  $[|\bot|] = \emptyset$\\
		By definition the statisfaction of a WFF of CTL is defined if for all $s \in S$, $\mu, s \not\models \bot$. Then no $s \in S$ can model  falisty. Then the set of states the model falsity is empty.
	\item $[|\neg \phi|] = S - [|\phi|]$\\
		By definition $\mu, s \models \neg \phi$ if and only if $\mu, s \not\models \phi$. So if you take all the $s \in S$ that models $\phi$  and remove them from the set of all possible states, you will be left with all the states that do not model $\phi$ and therefore must model $\neg \phi$
	\item $[|\phi_1 \land \phi_2|] = [|\phi_1|] \cap [|\phi_2|]$\\
	By definition  $\mu, s \models \phi \land \psi$ if and only if $\mu, s \models \phi$ and $\mu, s \models \psi$.
	So look at the set of all states that model $\phi$. Then also look at the set of states that model $\psi$, the intersection of these two sets will then contain the set of states such that $\mu, s \models \phi$ and $\mu, s \models \psi$ and therefore by definition will be the set of states that satisfies $\mu, s \models \phi \land \psi$ 
	\item $[|\phi_1 \lor \phi_2|] = [|\phi_1|] \cup [|\phi_2|]$\\
	By definition  $\mu, s \models \phi \lor \psi$ if and only if $\mu, s \models \phi$ or $\mu, s \models \psi$. So look at the set of all states that model $\phi$. Then also look at the set of states that model $\psi$, the union of these two sets will then contain the set of states such that $\mu, s \models \phi$ or $\mu, s \models \psi$ and therefore by definition will be the set of states that satisfies $\mu, s \models \phi \lor \psi$ 
	
	\item $[|\phi_1 \to \phi_2|] = (S- [|\phi_1|]) \cup [|\phi_2|]$\\
	 $[|\phi_1 \to \phi_2|] \equiv [|\neg \phi_1 \lor \phi_2|]$. So by the cases above we know  that $[|\neg \phi_1 \lor \phi_2|] =  [| \neg \phi_1|] \cup [|\phi_2|] =  (S - [|\phi_1|]) \cup [|\phi_2|] $
	 
	\item $[| \bold{AX} \phi|] = S - [|\bold{EX} \neg \phi|]  $\\
	Fact: $\bold{AX} \phi$ is equivelent to $\neg  \bold{EX} \neg \phi$. Therefore we can reomve from S all s such that $M,s \models \bold{EX} \neg \phi$ and are left with $S-[|\bold{EX}\neg \phi|]$
	
	\end{enumerate}
\end{document}

