%% to produce a PDF copy, issue the following command:
%%
%%     pdflatex propositional-logic-examples.tex
%%
%% in the same directory containing the LaTeX style files:
%%
%%     prooftree.sty  and  boxproof.sty

\documentclass[11pt,leqno,fleqn]{article}

\newcommand{\tab}{\hspace*{2em}}
\usepackage{amsfonts}
\usepackage{fullpage, enumerate}
\usepackage{amsmath}
\usepackage{amssymb}
\usepackage{qtree}
\usepackage{listings}

\usepackage{tikz}
\usetikzlibrary{arrows}
\usetikzlibrary{automata}

\usepackage{graphicx} 
\usepackage{times}              % better fonts for mathematical symbols
\usepackage{bm}                 % unlike \boldmath,
                                % \bm can be used anywhere within math mode
\usepackage[scaled=0.9]{helvet} % makes text a little smaller throughout,
                                % but not the text in math mode.
%\usepackage{./tex/latex/misc/prooftree}
%\usepackage{./tex/latex/misc/boxproof}

\setlength\hoffset{-5pt}      % horizontal offset, to move text horizontally
\setlength{\textwidth}{4.5in} % try different widths
\setlength\voffset{-5pt}      % vertical offset, to move text vertically
\setlength{\textheight}{7in}  % try different heights

\newcommand{\Hide}[1]{}             % use \Hide{bla bla} to hide ``bla bla''
\newcommand{\code}[1]{\texttt{#1}}  % use \code{...} to produce ASCII chars
\newcommand{\Intro}[1]{{#1}{\textrm{i}}}
\newcommand{\Elim}[1]{{#1}{\textrm{e}}}


\newcommand{\TTT}{\bm{\mathsf{T}}}
\newcommand{\FFF}{\bm{\mathsf{F}}}

\title{CS 512, Spring 2014
       \\[1ex]
       \textbf{Assignment 7}}
\author{Shan Sikdar} 
\date{} % omit date

\begin{document}

\maketitle

\section{Problem 1.3.48 a,b,c,d}
\begin{enumerate}[(a)]
\item  $\bold{AF}q$\\
$s_0$: Yes: satisfies since q is true at $s_0$.\\
$s_2$: Yes: From $s_2$ you can only go to $s_0$ and $s_3$ where q is true
\item $\bold{AG}(\bold{EF}(p \lor r))$\\
$s_0$: Yes: since in every  state p or r is true\\
$s_2$: Yes: since in every state p or r is true
\item $\bold{EX}(\bold{EX}r)$\\
$s_0$: Yes: look at the path $s_0 s_1 s_1...$\\
$s_2$: Yes: look at the path $s_2 s_0 s_1...$
\item $\bold{AG}(\bold{AF}q)$\\
$s_0$: No: since we can have the path $s_0 s_1 s_1....$\\
$s_2$: No: since we can have the path $s_2 s_0 s_1...$
\end{enumerate}

\newpage
\section{Problem 3.4.9}
new $\bold{AG} \phi$: \tab $\bold{AX}(\bold{AG} \phi)$\\
new $\bold{EG} \phi$: \tab $\bold{EX}(\bold{EG}\phi)$\\
new $\bold{AF} \phi$': \tab$\bold{AXA}\phi$\\
new $\bold{EF} \phi$: \tab$\bold{EXEF}\phi$\\
new $\bold{A}[\phi_1 \bold{U} \phi_2]$: \tab $\phi_1 \land \bold{AXA}[\phi_1 \bold{U} \phi_2]$\\
new $\bold{A}[\phi_1 \bold{U} \phi_2]$: \tab $\phi_1 \land \bold{EXE}[\phi_1 \bold{U} \phi_2]$

\section{Problem 3.4.10 a,b,c,d,e}
\begin{enumerate}[(a)]

\item $\bold{EF}\phi$ and $\bold{EG}\phi$\\
In the state transition system below note that: $\mathcal{M},s_0 \models \bold{EF}\phi $ but not   $\bold{EG}\phi$. So they are not equivelent.\\
\begin{tikzpicture}[->,>=stealth',shorten >=1pt,auto,node distance=3cm,
                    thick,main node/.style={circle,draw,font=\sffamily\Large\bfseries}]

  \node[main node] (1) {$s_0: \neg p$};
  \node[main node] (2) [ right of=1] {$s_1: p$};
  \node[main node] (4) [ right of=2] {$s_2: \neg p$};
\draw[<-] (1) -- node[above] {start} ++(-2cm,0);
  \path[every node/.style={font=\sffamily\small}]
    (1) 
        edge [ right] node[left] {} (2)
    (2) 
        edge node {} (4)

    (4) edge [loop right] node {} (4);
\end{tikzpicture}

\item $\bold{EF}\phi \lor \bold{EF}\psi$ and $\bold{EF}(\phi \lor \psi)$\\
They are equivelent.\\
1. First, assume that $s \models \bold{EF}\phi \lor \bold{EF}\psi$. Without loss of generality, assume that $s \models \bold{EF}\phi$. This means that there is a future state $s_n$, reachable from s, such that $s_n \models \phi$. But then $s_n \models \phi \lor \psi$ follows. But this means that there is a state rechable from s which satisfies $\phi \lor \psi$. Thus, $s \models \bold{EF}(\phi \lor \psi)$ follows.\\
2. Second, assume that $s \models \bold{EF}(\phi \lor \psi)$. Then there exists a state $s_m \models \phi \lor \psi$Withou loss of generality, we may assume that $s_m \models \psi$. Buth then we can conclude that $s \models \bold{EF} \psi$, as $s_m$ is reachable from s. Therefore, we also have $s \models \bold{EF} \phi \lor \bold{EF}\psi$
\item  $\bold{AF}\phi \lor \bold{AF} \psi$ and $\bold{AF}(\phi \lor \psi)$\\
In the state transition system below, look at paths $\pi_1 = s_0 s_1 s_1...$ and $\pi_2 = s_0 s_2 s_2 ....$. They both statisy  $\bold{AF}(\phi \lor \psi)$ but $\pi_1$ does not statisy $\bold{AF}\psi$ and $\pi_2$ does not statisfy $\bold{AF}\phi$  and therefore $\bold{AF}\phi \lor \bold{AF} \psi$ cannot hold  \\
\begin{tikzpicture}[->,>=stealth',shorten >=1pt,auto,node distance=3cm,
                    thick,main node/.style={circle,draw,font=\sffamily\Large\bfseries}]


  \node[main node] (1) {$s_0$};
  \node[main node] (2) [below left of=1] {$s_1: \phi$};
  \node[main node] (4) [below right of=1] {$s_2: \psi$};
\draw[<-] (1) -- node[above] {start} ++(-2cm,0);
  \path[every node/.style={font=\sffamily\small}]
    (1) edge node [left] {} (4)
        edge node [right] {} (2)
    (2) 
        edge [bend right] node {} (4)
        edge [loop left] node {} (4)
    (4) 
        edge [loop right] node {} (4)
        edge [bend right] node[right] {} (2);
\end{tikzpicture}

\item $\bold{AF} \neg \phi$ and $\neg \bold{EG} \phi$\\
They are equivelent.\\
1. Assume $s \models  \bold{EF \neg \phi}$ Then at some state along this path we have $s_m \models \neg \phi$ Then  this path $s \not \models \bold{G} \phi$. Since was any arbitary path, then we know that $s \not \models EG \phi$. Then by definition we know $s \models \neg \bold{EG}\phi$.\\
2. Assume  $s \models \neg \bold{EG} \phi$ . Then we know at some point that $s_m \models \neg \phi$ So we know for all paths at some point will have $\neg \phi$. Therefore $s \models \bold{AF}\neg \phi$

\item $\bold{EF} \neg \phi$ and $\neg \bold{AF}\phi$\\
Look at the state transition diagram below. The path $s_0 s_1 s_1 ...$ satisfies  $\bold{EF} \neg \phi$ but not  $\neg \bold{AF}\phi$\\
\begin{tikzpicture}[->,>=stealth',shorten >=1pt,auto,node distance=3cm,
                    thick,main node/.style={circle,draw,font=\sffamily\Large\bfseries}]


  \node[main node] (1) {$s_0$};
  \node[main node] (2) [below left of=1] {$s_1: \phi$};
  \node[main node] (4) [below right of=1] {$s_2: \psi$};
\draw[<-] (1) -- node[above] {start} ++(-2cm,0);
  \path[every node/.style={font=\sffamily\small}]
    (1) edge node [left] {} (4)
        edge node [right] {} (2)
    (2) 
        edge [bend right] node {} (4)
        edge [loop left] node {} (4)
    (4) 
        edge [loop right] node {} (4)
        edge [bend right] node[right] {} (2);
\end{tikzpicture}
\end{enumerate}

\newpage
\section{3.5.6 a,b,c}
\begin{enumerate}[(a)]
\item $\bold{AFG}p$ and $\bold{AFAG}p$\\
Look at the transition digram below $\mathcal{M},s_0 \models \bold{AFG}p$ but not $\bold{AFAG}p$.\\
\begin{tikzpicture}[->,>=stealth',shorten >=1pt,auto,node distance=3cm,
                    thick,main node/.style={circle,draw,font=\sffamily\Large\bfseries}]


  \node[main node] (1) {$s_0$};
  \node[main node] (2) [below left of=1] {$s_1$};
  \node[main node] (4) [below right of=1] {$s_2: p$};
\draw[<-] (1) -- node[above] {start} ++(-2cm,0);
  \path[every node/.style={font=\sffamily\small}]
    (1) edge node [left] {} (4)
        edge node [right] {} (2)
    (2) 
        edge [bend right] node {} (4)
    
    (4) 
        edge [loop right] node {} (4)
        edge [bend right] node[right] {} (2);
\end{tikzpicture}
\item $\bold{AGF}p$ and $\bold{AGEF}p$\\
In the state transition system below note that: $\mathcal{M},s_0 \models \bold{AGEF}p$ but not   $\bold{AGF}p$. So they are not equivelent.\\
\begin{tikzpicture}[->,>=stealth',shorten >=1pt,auto,node distance=3cm,
                    thick,main node/.style={circle,draw,font=\sffamily\Large\bfseries}]

  \node[main node] (1) {$s_0: \neg p$};
  \node[main node] (2) [ right of=1] {$s_1: p$};
  \node[main node] (4) [ right of=2] {$s_2: \neg p$};
\draw[<-] (1) -- node[above] {start} ++(-2cm,0);
  \path[every node/.style={font=\sffamily\small}]
    (1) 
        edge [ right] node[left] {} (2)
    (2) 
        edge node {} (4)

    (4) edge [loop right] node {} (4);
\end{tikzpicture}
\item $\bold{A}[(p \bold{U}r) \lor (q \bold{U} r)]$ and $\bold{A}[(p \lor q) \bold{U} r]$\\
Look at the transition diagram above. note that:\\
 $\mathcal{M},s_0 \models \bold{A}[(p \lor q) \bold{U} r]$ but not $\bold{A}[(p \bold{U}r) \lor (q \bold{U} r)]$. so they are not equivelent.\\
\begin{tikzpicture}[->,>=stealth',shorten >=1pt,auto,node distance=3cm,
                    thick,main node/.style={circle,draw,font=\sffamily\Large\bfseries}]

  \node[main node] (1) {$s_0:p$};
  \node[main node] (2) [ right of=1] {$s_1: q$};
    \node[main node] (3) [ right of=2] {$s_2: p$};
  \node[main node] (4) [ right of=3] {$s_3:r$};
\draw[<-] (1) -- node[above] {start} ++(-2cm,0);
  \path[every node/.style={font=\sffamily\small}]
    (1) 
        edge [ right] node[left] {} (2)
    (2) 
        edge node {} (3)
    (3) edge node {} (4)
    (4) edge [loop right] node {} (4);
\end{tikzpicture}


\end{enumerate}


\end{document}

